\documentclass[12 pt, a4paper]{book}

% Font and language settings
\usepackage[T1]{fontenc}
\usepackage[polish]{babel}
\usepackage[utf8]{inputenc}
\usepackage{indentfirst}
\usepackage{csquotes}

% font types

% bibliography
\usepackage{biblatex}
\addbibresource{pagacz_TEX_bibliography.bib}

\begin{document}
\part{Księga pierwsza}
\chapter{Algebra Boole'a}
\section{Uwagi wstępne, treść rozdziału}
Znamy obecnie różne, nieraz mocno od siebie odbiegające sposoby wprowadzania pojęcia 
prawdopodobieństwa: teoria ,,klasyczna'' w różnych odmianach, teorie aksjomatyczne:
 {\fontfamily{qcr}\selectfont Bohlmanna, Keynesa, Kołmogrowa}, teoria ,,częstościowa''
{\fontfamily{qcr}\selectfont Misesa} i inne \cite{bohlmann1}.

\end{document}