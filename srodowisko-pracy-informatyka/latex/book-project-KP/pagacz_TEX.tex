\documentclass[12 pt, a4paper, leqno]{book}

% Font and language settings
\usepackage[T1]{fontenc}
\usepackage[utf8]{inputenc}
\usepackage{polski}
\usepackage{indentfirst}
\usepackage{csquotes}
\usepackage{lmodern}

% tables environment
\usepackage{array}

% bibliography
\usepackage[style=verbose, backend=biber]{biblatex}
\addbibresource{pagacz_TEX_bibliography.bib}
\DefineBibliographyStrings{english}{pages = {str\adddot}}


% unfortunate section, chapters and part numbering
\usepackage{titlesec}

\newcommand{\polishordinal}[1]{
    \ifcase\value{#1}\or 
    PIERWSZA \or 
    DRUGA \or 
    TRZECIA \or 
    CZWARTA
    \fi 
}

\titleclass{\part}{top}
\titleformat{\part}[display]
  {\normalfont\centering\bfseries}
  {\scriptsize{KSIĘGA \polishordinal{part}}}
  {3pt}
  {\large\MakeUppercase}
\titlespacing{\part}{0pt}{-1.5in}{1.2in}

\renewcommand{\thechapter}{\Roman{chapter}}
\titleclass{\chapter}{straight}
\titleformat{\chapter}[display]{\normalfont\centering}{\chaptertitlename\ \thechapter}{15pt}{\bfseries\large}
\titlespacing{\chapter}{0pt}{0pt}{15pt}

\renewcommand{\thesection}{\arabic{section}}
\titleclass{\section}{straight}
\titleformat{\section}{\normalfont\centering\bfseries}{\S\,\thesection.}{5pt}{}
\titlespacing{\section}{0pt}{10pt}{10pt}

% mathematic symbols and equations
\usepackage{amssymb, amsmath, multicol}
\numberwithin{equation}{section}

% list environments
\usepackage{enumitem}

% footnote counter reset every page and the weird footnote formatting
\usepackage[perpage]{footmisc}
\renewcommand{\thefootnote}{\textsuperscript{\arabic{footnote}})}
\makeatletter
\renewcommand{\@makefnmark}{\normalfont\@thefnmark}
\renewcommand{\@makefntext}[1]{
    \makebox[30pt][r]{\footnotesize\@makefnmark} #1
    }
\makeatother

% ref command for Boole definition
\newcommand{\refb}[1]{(\ref{#1})}

% new command for a very weird Dowód
\newcommand{\dowod}{D\,o\,w\,ó\,d}

% for italics in math mode
\newcommand{\italics}[1]{\mbox{\textit{#1}}}

% environment Twierdzenie, Dowód oraz Wniosek
\usepackage{amsthm}

\newtheoremstyle{twr}% name of the style to be used
  {}% measure of space to leave above the theorem. E.g.: 3pt
  {}% measure of space to leave below the theorem. E.g.: 3pt
  {\itshape}% name of font to use in the body of the theorem
  {\parindent}% measure of space to indent
  {\scshape}% name of head font
  {.}% punctuation between head and body
  { }% space after theorem head; " " = normal interword space
  {}% Manually specify head

\newtheoremstyle{eqtext}% name of the style to be used
  {}% measure of space to leave above the theorem. E.g.: 3pt
  {}% measure of space to leave below the theorem. E.g.: 3pt
  {}% name of font to use in the body of the theorem
  {0pt}% measure of space to indent
  {}% name of head font
  {}% punctuation between head and body
  {\parindent}% space after theorem head; " " = normal interword space
  {(#2)}% Manually specify head

\newtheoremstyle{okreslenie}% name of the style to be used
  {}% measure of space to leave above the theorem. E.g.: 3pt
  {}% measure of space to leave below the theorem. E.g.: 3pt
  {}% name of font to use in the body of the theorem
  {\parindent}% measure of space to indent
  {\scshape}% name of head font
  {.}% punctuation between head and body
  { }% space after theorem head; " " = normal interword space
  {}% Manually specify head

\theoremstyle{twr}
\newtheorem{twr}{Twierdzenie}

\theoremstyle{twr}
\newtheorem*{wniosek}{Wniosek}

\theoremstyle{eqtext}
\newtheorem{eqtext}[equation]{Equation}

\theoremstyle{okreslenie}
\newtheorem{okreslenie}{Określenie}


% Headers and footers
\usepackage{fancyhdr}
\setlength{\headheight}{15.2pt}

\pagestyle{fancy}
\renewcommand{\chaptermark}[1]{ \markboth{\footnotesize \thechapter . #1}{} } % changing the content of the leftmark
\renewcommand{\sectionmark}[1]{ \markright{\footnotesize \S\ \thesection .\ #1}{} }  % changing the content of the rightmark
\renewcommand{\headrulewidth}{0pt}
\lhead[]{\thepage} % [odd output] {even output}
\chead[\rightmark]{\leftmark}
\rhead[\thepage]{}
\rfoot[\tiny Podstawy teorii prawdopodobieństwa]{}
\cfoot[]{}
\lfoot[]{\tiny Podstawy teorii prawdopodobieństwa}

\fancypagestyle{plain}{ % chapter pages - nothing in them
  \fancyhf{} 
}

% spacing around math display mode
\def\mycommand{\setlength{\abovedisplayskip}{7pt}%
\setlength{\belowdisplayskip}{7pt}%
\setlength{\abovedisplayshortskip}{7pt}%
\setlength{\belowdisplayshortskip}{7pt}}

\let\oldselectfont\selectfont
\def\selectfont{\oldselectfont\mycommand}

\mycommand

% spacing around enumerate lists
\setlist[enumerate]{itemsep=3pt,topsep=3pt}

% length of the footnote rule
\renewcommand{\footnoterule}{%
  \kern -3pt
  \hrule width 0.15\textwidth height 0.5pt
  \kern 2pt
}

% defining bibliography
\usepackage{filecontents}
\begin{filecontents}{pagacz_TEX_bibliography.bib}
  @book{
    bohlmann1,
    title = "Lebensversicherungsmethodik",
    author = "G. Bohlman",
    series = "Enzyklop{\"a}die der Mathematischen Wissenschaften, Bd.1, Teil II",
    location = "Leipzig",
    year = "1900-1904",
    pages = "857-917"
}

@book{
    keynes1,
    author = "J. M. Keynes",
    title = "A Treatise on Probability",
    location = "London and New York 1921",
    edition = "II",
    year = "1929"
}

@book{
    kolmogoroff1,
    author = "A. Kolmogoroff",
    title = "Grunbegriffe der Wahrscheinlichkeitsrechnung",
    series = "Ergebnisse der Mathematik II.3 (1933)"
}

@book{
    mises1,
    author = "R. Mises",
    title = "Wahrscheinlichkeitsrechnung und ihre Anwendung in der Statistik und theoretischen Physik",
    location = "Wien",
    year = "1931"
}

@book{
    tarski1,
    author = "A. Tarski",
    title = "O logice matematycznej i metodzie dedukcyjnej",
    location = "Lwów-Warszawa, w szczególności str. 85-89"
}

@book{
    huntington1,
    author = "E. V. Huntington",
    title = "Sets of independent postulates for the algebra of logic",
    location = "Transations of the American Mathematical Society 5 (1904)",
    pages = "288-309"
}

@book{
    birkhoff1,
    author = "G. Birkhoff",
    title = "Lattice Theory",
    location = "New York 1948, II wyd."
}
\end{filecontents}

%%%%%%%%%%%%%
% main body
%%%%%%%%%%%%%
\begin{document}

\part{Elementarna teoria prawdopodobieństwa}
\chapter{Algebra Boole'a}

% Pierwsza sekcja
\section{Uwagi wstępne, treść rozdziału} \label{sec:wstep}
Znamy obecnie różne, nieraz mocno od siebie odbiegające sposoby wprowadzania pojęcia 
prawdopodobieństwa: teoria ,,klasyczna'' w różnych odmianach, teorie aksjomatyczne:
 {\fontfamily{qcr}\selectfont Bohlmanna, Keynesa, Kołmogorowa}, teoria ,,częstościowa''
{\fontfamily{qcr}\selectfont Misesa} i inne \footnote{\cite{bohlmann1}. \par \hspace{20pt} \cite{keynes1}. \par \hspace{20pt} \cite{kolmogoroff1}. \par \hspace{20pt} \cite{mises1}.}
Porównanie tych sposobów prowadzi do następujących spostrzeżeń:

$1^\circ$ Prawdopodobieństwo bywa określane rozmaicie, zawsze jednak jest ono liczbą nieujemną,
            nie większą od jedności, przyporządkowaną pewnym przedmiotom.

$2^\circ$ Co do natury przedmiotów, którym zostaje przyporządkowane prawdopodobieństwo, panuje rozbieżność
            między poszczególnymi teoriami, w każdym jednak razie można uważać za ustalone, że przedmioty 
            te (zdarzenia, zdania, zbiory, cechy) tworzą tak zwane \textit{ciała Boole'a}, którą
            intuicyjnie określić można jako algebrę wyrazów: ,,nie'', ,,i'' oraz ,,lub''.

Wobec tego podajemy w tym rozdziale zarys teorii ciał Boole'a, a więc: ich określenie
przez postulaty (\S\ \ref{sec:postulaty}), elementarne twierdzenia algebry Boole'a (\S\S\ \ref{sec:dwoistosc} i \ref{sec:elementarne}), związek teorii ciał Boole'a
z teorią zbiorów częściowo uporządkowanych (\S\ \ref{sec:zbiory}), określenie działań nieskończonych w ciałach Boole'a i prawa nimi rządzące (\S\ 6),
zastosowania ciał Boole'a w teorii zbiorów (\S\S\ 7 i 9) i w logice (\S\ 11), określenie podciał (\S\ 8), 
określenie homomorfizmu między dwoma ciałami Boole'a, kongruencji w ciałach Boole'a i uogólnienie ciał Boole'a na tak zwane
ciała zrelatywizowane (\S\ 10) oraz określenie atomu, ciała atomowego i operacji scalania
atomów (\S\S\ 12, 13). Ostatni paragraf każdego rozdziału będzie zawierał przykłady i zadania.

% Druga sekcja
\section{Określenie ciał Boole'a} \label{sec:postulaty}
\textit{Ciałem Boole'a} nazywamy zbiór $U$, na którego elementach określone są działania 
+, $\cdot$ i $'$ (dwa pierwsze na dwu, trzecie na jednym elemencie) w taki sposób, aby dla 
dowolnych elementów $u, v, w \in U$ były spełnione następujące związki:


\noindent
% \begin{small}
\begin{minipage}{0.55\textwidth}
\begin{flalign*} 
  & u + v \in U, \quad u \cdot v \in U, \quad u' \in U, & \tag{B1} \label{eq:b1} \\
  & u + (v + w) = (u+v)+w,  & \tag{B$^+$2} \label{eq:bp2} \\
  & u + v = v + u,  & \tag{B$^+$3} \label{eq:bp3} \\
  & u \cdot (v + w) = (u \cdot v) + (u \cdot w), & \tag{B$^+$4} \label{eq:bp4} \\
  & u + (u \cdot u') = u, & \tag{B$^+$5} \label{eq:bp5} \\
  & u + u' = v + v', & \tag{B$^+$6} \label{eq:bp6} \\
\end{flalign*}
\end{minipage}
\hspace{-20pt}
\begin{minipage}{0.53\textwidth}
\begin{flalign*}
  && \\
  & u \cdot (v \cdot w) = (u \cdot v) \cdot w, & \tag{B$^.$2} \label{eq:bd2} \\
  & u \cdot v = v \cdot u, & \tag{B$^.$3} \label{eq:bd3} \\
  & u + (v \cdot w) = (u + w) \cdot (u + w), & \tag{B$^.$4} \label{eq:bd4} \\
  & u \cdot (u + u') = u, & \tag{B$^.$5} \label{eq:bd5} \\
  & u \cdot u' = v \cdot v'. & \tag{B$^.$6} \label{eq:bd6} \\
\end{flalign*}
\end{minipage}
% \end{small}

Działanie $+$ nazywamy \textit{dodawaniem} w ciele $U$, a element $u+v$ \textit{sumą} elementów $u$ i $v$;
działanie $\cdot$ nazywamy \textit{mnożeniem} w ciele $U$, a element $u \cdot v$ \textit{iloczynem}
elementów $u$ i $v$, wreszcie działanie $'$ nazywamy \textit{dopełnianiem} w ciele $U$, a element $u'$ 
\textit{uzupełnieniem} elementu $u$.

Wyrażenie zapisane sensownie za pomocą zmiennych przebiegających zbiór $U$ lub nazw elementów zbioru $U$, znaków $+$, $\cdot$, 
$'$ i nawiasów nazywamy \textit{wyrażeniem algebraicznym} ciała $U$. Zdanie powstałe z dwu wyrażeń algebraicznych
przez połączenie ich znakiem identyczności ($+$) nazywamy \textit{równością algebraiczną} lub \textit{równością}.
Przy pisaniu wyrażeń algebraicznych umawiamy się, w celu oszczędzania pisania zbędnych nawiasów, 
że znak mnożenia ($\cdot$) wiąże silniej niż znak oddawania ($+$), a następnie umawiamy się opuszczać nawiasy 
przy wielokrotnych sumach lub iloczynach ze względu na aksjomaty \refb{eq:bp2} i \refb{eq:bd2} wyrażające łączność tych działań.
Ponadto będziemy opuszczali, ilekrośc nie będzie to groziło nieporozumieniem, znak iloczynu, pisząc zamiast $u \cdot v$
po prostu $uv$. To postępowanie upodabnia nasze znakowanie do znakowania zwykłej algebry, w której te uproszczenia 
są ogólnie przyjęte.

W ten sposób wyrażenie algebraczine $u +(v \cdot w)$ (por. \refb{eq:bp4}) zapisujemy w postaci $u+vw$, wyrażenie zaś
$u+[(v \cdot w)+(u \cdot w)]$ - w postaci $u + vw + uw$.

Związki \refb{eq:b1}-\refb{eq:bd6} nazywamy \textit{układem postulatów algebry Boole'a}, w skróceniu: \textit{układem} (B).
Wychodząc z postulatów układu (B) możemy, za pomocą rozumowań opartych na prawach logiki, wyprowadzać nowe związki prawdziwe w każdym ciele Boole'a, to znaczy spełnione
przez każdy układ elementów każdego ciała Boole'a. Ogół tych wyszstkich związków (zdań), które dają się wyprowadzić
z postulatów układu (B) przez logicznie poprawne rozumowanie, nazywamy \textit{systemem algebry Boole'a}.

Używając współczesnej terminologii logicznej możemy powiedzieć, że ciała Boole'a \footnote{\ Przykłady ciał Boole'a znajdzie czytelnik w \S\S\ 7 i 9 tego rozdziału.} są 
\textit{modelami systemu algebry Boole'a} lub - co na jedno wychodzi - \textit{modelami układu postulatów} (B) \footnote{\ Por. \cite{tarski1}.}.

Przedstawiony tu układ postulatów algenry Boole'a nie jest ani jedynym możliwym (tę własność dzieli on
ze wszystkimi układami postulatów abstrakcyjnych teorii), ani też najprostszym. Znane są rozmaite równoważne
między sobą układy postulatów algebry Boole'a \footnote{\ Różne układy postulatów algebry Boole'a bada \cite{huntington1}}. Zaierają one przeważnie mniej postulatów niż
podany tu układ (B), a często też mniej pojęć pierwotnych. Jednak podany tu układ postulatów ma wiele stron dodatnich;
poszczególne postulaty mają łatwo uchwytny sens intuicyjny, są łatwe do zapamiętania i elementarne twierdzenia dają się z nich 
prosto wyprowadzić. Poza tym uwidacznia on istotną dla ciał Boole'a symetrię między działaniami dodawania i mnożenia. 
W następnym paragrafie wyciągniemy z tej uwagi ważną konsekwencję.

Można wykazać, że do ugruntowania algebry Boole'a wystarczają postulaty \refb{eq:b1}, \refb{eq:bp3}, \refb{eq:bp4},
\refb{eq:bd4}, \refb{eq:bd5}, \refb{eq:bp6}, \refb{eq:bd6}, których zespół oznaczmy przez (B*); natomiast pozostałe
postulaty, tj. \refb{eq:bp2}, \refb{eq:bd2}, \refb{eq:bd3}, \refb{eq:bd5}, można z poprzednich wyprowadzić na drodze poprawnych
rozumowań. Można też wykazać, ze z układu postulatów (B*) nie da się już odrzucić żadnego postulatu bez uszczerbku dla systemu algebry Boole'a;
mówimy, że układ postulatów (B*) jest niezależny, a układ (B) nie jest taki.

% Trzecia sekcja
\sectionmark{Omówienie postulatów układu (B) -- Twierdzenie o dwoistości}
\section{Omówienie postulatów układu (B). Twierdzenie o dwoistości} \label{sec:dwoistosc}
\sectionmark{Omówienie postulatów układu (B) -- Twierdzenie o dwoistości}
Postulat \refb{eq:b1} ma nieco odmienny charakter od pozostałych. Żąda on, aby działania
$'$, $+$ i $\cdot$ były \textit{wykonalne} w ciele $U$, czyli aby ciało $U$ było \textit{zamknięte}
ze względu na podstawowe działania, to znaczy, aby element powstały w wyniku wykonania któregoś
 z działań podstawowych na elementach lub elemencie ciała $U$ należał do ciała $U$. Dalsze
 postulaty charakteryzują pewne właśności działań podstawowych, a mianowicie:
 postulaty \refb{eq:bp2} i \refb{eq:bd2} wyrażają \textit{łączność} działań dodawania i mnożenia,
 postulaty \refb{eq:bp3} i \refb{eq:bd3} ich \textit{przemienność}, postulaty \refb{eq:bp4} i \refb{eq:bd4}
 ich wzajemną \textit{rozdzielność}. Pozostałe cztery postulaty charakteryzują własności
 działania uzupełniania w związku z dodawaniem i mnożeniem.

 Postulaty \refb{eq:bp6} i \refb{eq:bd6} żądają, aby elementy przedstawione wyrażeniami algebraicznymi $u+u'$ i $uu'$
 nie zależały od wyboru elementu $u$ w ciele $U$. Są to więc dwa wyróżnione elementy w ciele $U$, które umawiamy się oznaczać
 odpowiednio przez 1 i 0, to znaczy przyjmujemy następujące określenia:
 \begin{align}
  1 &= u + u',  \label{eq:jedynka} \\ 
  0 &= uu'.  \label{eq:zero}
 \end{align}
 Mamy prawo do przyjęcia takich określeń, gdyż ich jednoznaczność gwarantują nam postulaty 
 \refb{eq:bp6} i \refb{eq:bd6}. Nie możemy co prawda twierdzić, że 0 nie jes tidentyczne z 1,
 nie wiemy bowiem, czy zbiór $U$ składa się z więcej niż jednego elementu, łatwo jednak dowieść,
 że jeżeli zbiór $U$ zawiera więcej niż jeden element, to 0 nie jest identyczne z 1.

 Niech bowiem będzie $0 = 1 \in U$ i $u \in U$ niech będzie elementem różnym od 1 (a więc i od 0).
 Ponieważ jak udowodnimy później (\S \ref{sec:elementarne}, \eqref{eq:taut}), w każdym ciele 
 Boole'a zachodzi równość $u=u+u$ dla dowolnego elementu, więc
\begin{enumerate}[label=(\Roman{*}), wide=0pt, widest=99,leftmargin=*, labelsep=15pt, itemsep=0pt]
  \item $uu' = u + u'$ (na mocy założenia $0=1$), 
  \item $u+(u+u')=u$ (na mocy postulatu \refb{eq:bp5} oraz (\RN{1})),
  \item $(u+u)+u'=u$ (na mocy postulatu \refb{eq:bp2} oraz (\RN{2})),
  \item $u+u'=u$ (na mocy równości $u+u=u$),
  \item $1=u$ (na mocy (\RN{4}) i określenia \refb{eq:jedynka})
\end{enumerate}

Równość (\RN{5}) przeczy założeniu, że $u$ jest elementem różnym od 1, a to dowodzi słuszności tezy.

Zwróćmy uwagę na to że elementy 0 i 1 zależą od ciała $U$, to znaczy, że jeżeli $U$ i $V$ są różnymi ciałami
Boole'a, to elementy 0 i 1 w ciele $U$ są na ogół różne od elementów o i 1 w ciele $V$.

Po przyjęciu określeń \refb{eq:jedynka} i \refb{eq:zero} możemy postulaty \refb{eq:bp5} i
\refb{eq:bd5} zapisać w postaci
\begin{displaymath}
\begin{array}{@{}w{l}{0.1\textwidth} wr{0.40\textwidth} @{\hspace{5pt}} c @{\hspace{5pt}} wl{0.40\textwidth}}
$\overline{(\mbox{B}^+5)}$ & $u + 0$ & $=$ & $u,$ \\[\medskipamount]
$\overline{(\mbox{B}^\cdot5)}$ & $u \cdot 1$ & $=$ & $u.$
\end{array}
\end{displaymath}

W tej postaci są one zwykle przyjmowane w układach postulatów algebry Boole'a.

Nie należy zapominać, że ciało Boole'a tworzy nie sam zbiór $U$, lecz zbiór $U$ wraz z określonymi 
w nim działaniami podstawowymi. Na przykład, gdy w pewnym zbiorze mamy określone dwa różne układy działań tworzące 
wraz z tym zbiorem ciało Boole'a, uważamy, że mamy do czynienia z dwoma różnymi caiłami Boole'a.

Z tego powodu byłoby rzeczą słuszną nazywać ciałem Boole'a nei sam zbiór $U$, lecz czwórkę uporządkowaną
$\langle U, +, \cdot, ' \rangle$, składającą się ze zbioru $U$ i trech działań podstawowych spełniających postulaty układu (B).
Nie będziemy się jednak trzymali tego sposobu oznaczania, gdyż mimo jego wielkiej ścisłości, nie jest on wygodny w użyciu.

\begin{twr}[o dwoistości]
Jeżeli zbiór $U$ jest ciałem Boole'a ze względu na działania $+, \cdot$ i $'$, to zbiór $U$ jest również ciałem Boole'a
ze względu na działania $\cdot, +$ i $'$.
\end{twr}

Twierdzenie to mówi nam, że każdy zbiór, który jest ciałem Boole'a ze względu na pewien układ postawowych działań, jest również ciałem Boole'a
ze względu na inny, różny od poprzedniego układu działań, działanie bowiem $\cdot$ nie pokrywa się z działaniem $+$ (z wyjątkiem trywialnego
przypadku, gdy ciało Boole'a jest zbiorem jednoelementowym).

Mówi nam ono dalej, że działanie $+$ i $\cdot$ określone w pewnym ciele Boole'a można traktować dowolnie: $+$ jako dodawanie, a $\cdot$ jako 
mnożenie, lub przeciwnie. Dowów twierdzenia 1 jest natychmiastowy. Jak już mówiliśmy, postulaty układu (B) wykazują symetrię, która została 
specjalnie zaznaczona w ich numeracji. Każdy z postulatów przechodzi w drugi oznaczony tym samym numerem, jeżeli w miejsce znaków
$+$ wpiszemy znaki $\cdot$ i odwrotnie. Jeżeli więc dwa działania $+$ i $\cdot$ (wraz z trzecim działaniem $'$, które nie ulega zmianie)
spełniają jeden z postulatów, to działania $\cdot$ i $+$ spełniają drugi z postulatów, po przestawieniu w nim znaków sumy i iloczynu.

Mając dane wyrażenie algebraiczne nazwiemy \textit{dwoistym} do niego wyrażenie algebraiczne otrzymane z danego w ten sposób, że występujące
w nim znaki iloczynu ($\cdot$) zastępujemy znakami ($+$) i na odwrót - znaki sumy znakami ilocynu (równocześnie zmieniając odpowiednio nawiasy, stosownie
do zawartych poprzednio umów dotyczących zapisywania wyrażeń algebraicznych). Ze względu na określenie \eqref{eq:jedynka} i \eqref{eq:zero} oraz fakt,
że wyrażenia $u + u' \quad uu'$ są względem siebie dwoiste, widzimy, że w przypadku występowania 0 lub 1 w danym wyrażeniu musimy, aby otrzymać wyrażenie względem
niego dwoiste, zastąpić w nim wszędzie 0 przez 1 i odwrotnie.

\textit{Równością algebraiczną dwoistą} względem danej nazywamy równość, której obie strony są wyrażeniami dwoistymi odpowiednio względem obu stron danej równości.
Widzimy, że każdy z postulatów układu (B) oznaczony krzyżykiem jest dwoisty wwzględem postulatu oznaczego tym samym numerem i kropką. Dalej widzimy,
że równości \eqref{eq:jedynka} i \eqref{eq:zero} są wzajemnie do siebie dwoiste. Wynika stąd następujący 

\begin{wniosek}
Jeżeli udowodnimy pewną równość algebraiczną opierając się w dowodzie kolejno na pewnych postulatach, to dowód równości dwoistej
otrzymamy zastępując kolejno w dowodzie danej równości postulaty, na których opieraliśmy się, postulatami względem nich dwoistymi.
\end{wniosek}

Korzystając z tego wniosku będziemy z dwóch równości dwoistych dowodzili tylko jednej; drugą będziemy uważali za udowodnioną na podstawie powyższego wniosku.
O słuszności tego postępowania może się czytelnik przekonać przeprowadzając w poszczególlnych pyrzypadkach dowód równości dwoistej na podstawie dowodu równości
udowodnionej.

% Czwarta sekcja
\section{Elementarne twierdzenia algebry Boole'a} \label{sec:elementarne}
Dowody równości algebraicznych będziemy pisali, podobnie jak się to robi w zwykłej algebrze, w postaci łańcuchów równości, zastępując przy przejściu
od członu do członu ,,równe przez równe'' na podstawie postulatów lub równości wcześniej udowodnionych.

Udowodnimy najpierw następujące twierdzenia dwoiste, zwane \textit{zasadami tautologii:}
\begin{equation} \label{eq:taut}
u = u + u, \quad u = u \cdot u
\end{equation}

\dowod. $u = u + uu' = (u + u)(u+u') = u + u.$ W dowodzie powołujemy się kolejno na postulaty \eqref{eq:bp5}, \eqref{eq:bd4} i \eqref{eq:bd5}.

Dla wykazania na przykładzie zastosowania wniosku wysnutego w poprzednim paragrafie podamy dowód twierdzenia dwoistego:
\begin{displaymath}
u = u \cdot (u + u') = uu + uu' = uu.
\end{displaymath}

Powołujemy się kolejno na postulaty \eqref{eq:bd5}, \eqref{eq:bp4} i \eqref{eq:bp5}.

Udowodnimy teraz twierdzenia:
\begin{equation} \label{eq:sumailoczynow}
u = uv + uv', \quad u = (u + v)(u + v').
\end{equation}

\dowod. $u = u(u + u') = u(v + v') = uv + uv'.$

Powołaliśmy się na postulaty \eqref{eq:bd5}, \eqref{eq:bp6} i \eqref{eq:bp4}.
\begin{equation} \label{eq:absorpcja}
u = u + uv, \quad u = u(u + v).
\end{equation}

\dowod.
\begin{displaymath}
u = uv + uv' = uv' + uv = uv' + (uv + uv) = (uv' + uv) + uv = u + uv.
\end{displaymath}

Przy przechodzeniu od jednej równości do drugiej korzystaliśmy kolejno z wzorów \eqref{eq:sumailoczynow}, \eqref{eq:bp3}, \eqref{eq:taut},
\eqref{eq:bp2} i \eqref{eq:sumailoczynow}.

W dowodzie tego prawa, zwanego \textit{prawem absorpcji}, powołaliśmy się na postulat \eqref{eq:bp2} orzekający łączność działania $+$, po to,
aby pokazać jego zastosowanie.

Zgodnie z umową zawartą w \S 2 wielowyrazowe sumy i iloczyny bedziemy pisali bez nawiasów i nie będziemy się już więcej powoływali na postulaty 
\eqref{eq:bp2} i \eqref{eq:bd2}.

Oto dalsze dwa \textit{prawa absorpcji} oraz \textit{prawo podwójnego dopełnienia}:
\begin{equation} \label{eq:absorpcja2}
u + 1 = 1, \quad u \cdot 0 = 0.
\end{equation}

\dowod\ opiera się na wzorach \eqref{eq:jedynka}, \eqref{eq:taut} i \eqref{eq:jedynka}: $u + 1 = u + u + u' = u + u' = 1.$
\begin{equation} \label{eq:2przeczenie}
u = (u')'.
\end{equation}

Dowód opiera się na wzorach \eqref{eq:sumailoczynow}, \eqref{eq:bp3}, \eqref{eq:bd3}, \eqref{eq:bd6}, \eqref{eq:bd3}, \eqref{eq:sumailoczynow}:
\begin{gather*}
u = uu' + u(u')' = u(u')' + uu' = (u')'u + uu' = (u')'u + u'(u')' = \\
= (u')'u + (u')'u' = (u')'.
\end{gather*}

W dalszych dowodach nie będziemy zaznaczali miejsc, w których powołujemy się na postulaty przemienności \eqref{eq:bp3} i \eqref{eq:bd3}.

Udowodnimy teraz tak zwane wzory de M\,o\,r\,g\,a\,n\,a:
\begin{equation} \label{eq:morgan}
(u + v)' = u'v', \quad (uv)' = u' + v'.
\end{equation}

\dowod. Udowodnimy wpierw pomocniczo:
\begin{equation*} \tag{a}
  1 = u + v + u'v', \quad 0 = uv(u' + v')
\end{equation*}

Istotnie, $1 = (u + u')(v + v') = uv + uv' + u'v + u'v' = (uv + uv') + (uv + u'v) + u'v' = u + v + u'v'.$

Przy dowodzie tej równości opieraliśmy się kolejno na wzorach \eqref{eq:jedynka}, \eqref{eq:taut}, \eqref{eq:bp4}, \eqref{eq:taut}, \eqref{eq:sumailoczynow}.

Mamy teraz
\begin{gather*}
(u + v)' = (u + v + u'v')(u + v)' = (u + v)(u + v)' + u'v'(u + v)' = \\
= u'v'(u + v)' + u'v'\big((u')' + (v')'\big) = u'v'(u + v)' = u'v'(u+v) = \\
= u'v'[(u+v)' + (u + v)] = u'v',
\end{gather*}
przy czym oparliśmy się na wzorach \eqref{eq:bd5}, (a), \eqref{eq:bp4}, [\eqref{eq:bp5}, (a)], \eqref{eq:2przeczenie}, \eqref{eq:bp4}, \eqref{eq:bd5}.

\small{Umawiamy się, że gdy przy przejściu od jednej równości do drugiej powołujemy się na więcej niż jeden postulat lub poprzdnio udowodnioną równość, ujmujemy numery 
odpowiadających twierdzeń w nawiasy kwadratowe.}

\normalsize Udowodnimy teraz dalsze twierdzenia:
\begin{equation} \label{eq:odwrotnymorgan}
u + v = (u'v')', \quad uv = (u' + v')'.
\end{equation}

\dowod\ opieramy na wzorach \eqref{eq:2przeczenie}, \eqref{eq:morgan}: $u + v = [(u + v)']' = (u'v')'.$
\begin{equation} \label{eq:01przeczenia}
0 = 1', \quad 1 = 0'.
\end{equation}

\dowod\ opieramy na wzorach \eqref{eq:zero}, [\eqref{eq:odwrotnymorgan}, \eqref{eq:2przeczenie}], \eqref{eq:jedynka}: $0 = uu' = (u' + u)' = 1'.$

Określamy teraz indukcyjnie, podobnie jak w arytemtyce, sumę i iloczyn $n$-elementów $u_1, u_2, \ldots, u_n$ ciała $U$:
\begin{equation}
  \sum\limits_{i=1}^{1}u_i=u_1,\quad \prod\limits_{i=1}^{1}u_i=u_1.
\end{equation}
\begin{equation}
  \sum\limits_{i=1}^{n}u_i = \big(\!\sum\limits_{i=1}^{n-1}u_i\big) + u_n, \quad \prod\limits_{i=1}^{n}u_i = \big(\!\prod\limits_{i=1}^{n-1}u_i\big) + u_n, \quad \mbox{dla}\ n > 1.
\end{equation}

Można drogą łatwej indukcji uogólnić wzory de Morgana na sumy i iloczyny $n$-elementów:
\begin{equation}
  \big(\!\sum\limits_{i=1}^{n}u_i\big)' = \!\prod\limits_{i=1}^{n}u_i', \quad \big(\!\prod\limits_{i=1}^{n}u_i\big)' = \!\sum\limits_{i=1}^{n}u_i'
\end{equation}

Na zakończenie tego paragrafu podamy jeszcze kilka twierdzeń nie mających postaci równości.
\begin{eqtext}
\textit{Jeżeli} $u + v = 1$ \textit{i} $uv = 0,$ \textit{to} $u = v'.$
\end{eqtext}

\dowod. Załóżmy, że (a) $u + v = 1$, (b) uv = 0. Opierając się kolejno na wzorach [\eqref{eq:bp5}, \eqref{eq:bd6}], \eqref{eq:bd4}, [(a), \eqref{eq:jedynka}, \eqref{eq:bd5}],
[(b), \eqref{eq:morgan}, \eqref{eq:01przeczenia}, \eqref{eq:bd6}], \eqref{eq:sumailoczynow} otrzymujemy
\begin{displaymath}
  u = u + vv' = (u + v)(u + v') = u + v' = (v' + u)(v' + u') = v' + uu' = v'.
\end{displaymath}
\begin{eqtext} \label{eq:4rownosci}
\textit{Nastepujące czztery wzory są równoważne} (to znaczy, jeżeli którykolwiek z nich zachodzi dla dwu elementów $u,v \in U$, to zachodzą też i pozostałe);
\begin{displaymath}
\mbox{(a)}\ u + v = v, \quad \mbox{(b)}\ u \cdot v = u, \quad \mbox{(c)}\ u' + v = 1, \quad \mbox{(d)}\ u \cdot v' = 0.
\end{displaymath}
\end{eqtext}

\dowod. Załóżmy (a). Otrzymujemy $u \cdot v = u(u + v) = u$, czyli (b), przy czym oparliśmy się na wzorach (a) i \eqref{eq:absorpcja}.

Załóżmy (b). Otrzymujemy $u' + v = (uv)' + v = u' + v' + v = 1,$ czyli (c), przy czym oparliśmy się na wzorach (b), \eqref{eq:morgan}, [\eqref{eq:jedynka}, \eqref{eq:absorpcja2}].

Załóżmy (c). Otrzymujemy $u \cdot v' = (u' + v)' = 1' = 0,$  czyli (d), przy czym oparliśmy się na wzorach [\eqref{eq:morgan}, \eqref{eq:2przeczenie}], (c), \eqref{eq:01przeczenia}.

Załóżmy (d). Otrzymujemy $u + v = uv + uv' + v = uv + v = v,$ czyli (a), przy czym oparliśmy się na wzorach \eqref{eq:sumailoczynow}, (d) i \eqref{eq:absorpcja}.

Wykazaliśmy, że z (a) wynika (b), z (b) wynika (c), z (c) wynika (d) i z (d) wynika (a), co daje \eqref{eq:4rownosci}.
\begin{eqtext} \label{eq:sekwsum}
\textit{Jeżeli} $u + v = v$ \textit{i} $v + z = z,$ \textit{to} $u + z = z.$
\end{eqtext}

\dowod. Załóżmy (a) $u + v = v,$ (b) $v + z = z$; opierając się na wzorach (b), (a) i (b) \footnote{\ Pozornie w dowodzie tym nie korzystamy z żadnych specyficznych twierdzeń
algebry Boole'a, w istocie jednak korzystamy z \eqref{eq:bp2}, czego zgodnie z umową nie uwidaczniamy.} otrzymujemy $u + z = u + v + z = v + z = z.$
\begin{eqtext}
$u = v$ \textit{wtedy i tylko wtedy, gdy} $uv' + u'v = 0.$
\end{eqtext}

\dowod. Załóżmy (a) $u = v$; otrzymujemy $uv' + u' = uu' + u'u = 0$, gdzie skorzystaliśmy z wzorów [(a), (a)], [\eqref{eq:sumailoczynow}, \eqref{eq:taut}].

Załóżmy (b) $uv' + u'v = 0$; otrzymujemy $u = uv + uv' = uv + uv' + uv' +u'v = uv + uv' +u'v = uv + (uv' + u'v) + u'v = uv + u + v = v.$ Oparliśmy się na wzorach
\eqref{eq:sumailoczynow}, (b), \eqref{eq:taut}, \eqref{eq:taut}, (b) i \eqref{eq:sumailoczynow}.

% Sekcja piąta
\sectionmark{Zawieranie -- Ciała Boole'a jako zbiory częściowo uporządkowane}
\section{Zawierania (implikacja). Cała Boole'a jako zbiory częściowo uporządkowane} \label{sec:zbiory}

Określimy obecnie pewną relację dwuargumentową między elementami dowolnego ciała Boole'a, zwaną relacją \textit{zawierania} lub \textit{implikacji}.

\begin{okreslenie}
Jeżeli $u$ i $v$ są dwoma elementami ciała Boole'a, to mówimy, że \textit{element u jest zawarty w elemencie v} (lub że \textit{u implikuje v}), i piszemy $u\rightarrow v$, jeśli $u + v = v.$
\end{okreslenie}

Na mocy tego określenia i twierdzenia \eqref{eq:4rownosci} wzór $u\rightarrow v$ jest równoważny każdemu z wzorór \eqref{eq:4rownosci}: (a), (b), (c) i (d).

Następujące własności implikacji otrzymamy natychmiast z udowodnionych już twierdzeń:
\begin{equation} \label{eq:zwrotnosc}
u \rightarrow u.
\end{equation}

Jest to tak zwana \textit{zwrotność} stosunku zawierania. Dowód opiera się na wzorze \eqref{eq:taut}.
\begin{eqtext} \label{eq:przechodniosc}
\textit{Jeżeli} $u \rightarrow v$ \textit{i} $v \rightarrow z$, \textit{to} $u \rightarrow z$.
\end{eqtext}

Jest to tak zwana \textit{przechodniość} stosunku zawierania. Dowód opiera się na wzorze \eqref{eq:sekwsum}.
\begin{equation}
  u \rightarrow 1.
\end{equation}

\dowod\ wynika z wzoru \eqref{eq:absorpcja2}.
\begin{equation}
0 \rightarrow u.
\end{equation}

\dowod \ wynika z wzorów \eqref{eq:absorpcja2} i \eqref{eq:sekwsum}.
\begin{equation}
u \rightarrow u + v.
\end{equation}

\dowod \ wynika z twierdzeń \eqref{eq:absorpcja} i \eqref{eq:sekwsum}.
\begin{equation}
  uv \rightarrow u.
\end{equation}

\dowod \ wynika z wzoru \eqref{eq:absorpcja}.
\begin{eqtext}
\itshape Jeżeli $u \rightarrow z$ i  $v \rightarrow z,$ to  $u + v \rightarrow z.$
\end{eqtext}

\dowod. Załóżmy (a) $u + z = z,$ (b) $v + z = z$; otrzymujemy: $u + v + z = u + z = z,$ przy czym opieraliśmy się na wzorach (b), (a).
\begin{eqtext}
\itshape Jeżeli $z \rightarrow u$ i $z \rightarrow v$, to $z \rightarrow u\cdot v$.
\end{eqtext}

\dowod \ jest analogiczny do poprzedniego przy użyciu równoważności \eqref{eq:4rownosci}. Pozostawimy go czytelnikowi.
\begin{eqtext}
\itshape Jeżeli $u \rightarrow 0,$ to $u = 0$.
\end{eqtext}

\dowod. Załóżmy (a) $u + 0 = 0$; otrzymujemy $u = u + uu' = u + 0 = 0.$

Oparliśmy się na wzorach \eqref{eq:bp5}, \eqref{eq:zero}, (a).
\begin{eqtext}
\itshape Jeżeli $1 \rightarrow u$, to $u = 1.$
\end{eqtext}

\dowod, który jest analogiczny do poprzedniego, pozostawiamy czytelnikowi.
\begin{eqtext}
\itshape Jeżeli $u \rightarrow v$, to $v' \rightarrow u'.$
\end{eqtext}

\dowod. Załóżmy (a) $u + v = v$; opoierając się na wzorach (a) i \eqref{eq:morgan} otrzymujemy $v' = (u + v)' = u'v',$ co na mocy \eqref{eq:4rownosci}
jest równoważne wzorowi $v' + u' = u'.$
\begin{eqtext}
  \itshape $u = v$ wtedy i tylko wtedy, gdy $u \rightarrow v$ i $v \rightarrow u$.
\end{eqtext}

\dowod. Z założenia, że $u = v$, wynika na mocy \eqref{eq:zwrotnosc} $u \rightarrow v$ i $v \rightarrow u$. Załóżmy, że $u + v = v$ i $v + u = u$. Korzystając
z \eqref{eq:bp3} mamy stąd $u=v.$

Zajmiemy się teraz abstrakcyjnym badaniem takich relacyj, które mają własności podobne do relacji zawierania w ciałach Boole'a.

Niech $A$ będzie dowolnym zbiorem, $\prec$ relacją określoną dla elementów zbioru $A$. O relacji $\prec$ mówimy, że \textit{porządkuje częściowo} zbiór $A$, jeżeli
ma następujące dwie własności:
\begin{enumerate}[label=(\Roman{*}), wide=0pt, widest=99,leftmargin=*, labelsep=15pt, itemsep=0pt]
  \item \itshape Jeżeli $a \in A$, to $a \prec a$. \normalfont
  \item \itshape Jeżeli $a, b, c \in A,\ a \prec b$ i $b \prec c$, to $a \prec c$. \normalfont
\end{enumerate}

Własności (I) i (II) nazywamy odpowiednio \textit{zwrotnością} i \textit{przechodniością} relacji $\prec$ w zbiorze $A$. Jeżeli relacja $\prec$ ma ponadto własność 

\begin{enumerate}[label=(\Roman{*}), wide=0pt, widest=99,leftmargin=*, labelsep=15pt, itemsep=0pt]
\setcounter{enumi}{2}
  \item \itshape Jeżeli $a,b \in A,\ a \prec b$ i $b \prec a$, to $a = b$, \normalfont
\end{enumerate}
to mówimy, że jest ona w zbiorze $A$ \textit{zredukowana.}

W przypadku, gdy prócz warunków (I), (II), (III) relacja $\prec$ spełnia jeszcze warunki:
\begin{enumerate}[label=(\Roman{*}), wide=0pt, widest=99,leftmargin=*, labelsep=15pt, itemsep=0pt]
\setcounter{enumi}{3}
  \item \itshape Jeżeli $a, b \in A$, to istnieje takie $c \in A$, że $a \in c,\ b\prec c$ i dla każdego $d\in A$, jeżeli $a \prec d$ i $b\prec d$, to $c \prec d$, \normalfont
  \item \itshape Jeżeli $a, b \in A$, to istnieje takie $c \in A$, że $a \in c,\ b\prec c$ i dla każdego $d\in A$, jeżeli $d \prec a$ i $d \prec b$, to $d \prec c$, \normalfont 
\end{enumerate}
mówimy, że \textit{indukuje} ona w zbiorze $A$ \textit{strukturę}.

Tak określone pojęcie struktury \footnote{Teoria struktur jest szczegółowo omówiona w książce: \cite{birkhoff1}} ma wiele zastosowań w różnych działach matematyki. Czytelnik sprawdzi, że np. relacja
,,$k$ dzieli bez reszty $l\,$'' indukuje strukturę w zbiorze liczb całkowitych.

\end{document}