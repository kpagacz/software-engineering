\documentclass[12 pt, reqno]{article}

\author{Konrad Pagacz}

\usepackage[cp1250]{inputenc}

\usepackage{polski, amssymb, amsmath}

\begin{document}
Przypomnijmy kilka znanych wzor�w:
\begin{equation}
(a+b)^2=a^2+2ab+b^2 ; \label{plus2} %przyk�adowa etykieta
\end{equation}
\begin{equation}
(a+b+c)^2 = a^2 + b^2 + c^2 + 2ab + 2ac + 2bc ; \label{kwadrat-sumy-trzech}
\end{equation}
\begin{equation}
(a-b)^2=a^2-2ab+b^2 ; \label{minus2}
\end{equation}
\begin{equation}
a^2-b^2=(a+b)(a-b); \label{roznica2}
\end{equation}
\begin{equation}
a^3+b^3=(a+b)(a^2-ab+b^2). \nonumber
\end{equation}

Ostatni wz�r zosta� zapisany przy u�yciu �rodowiska \texttt{equation}, ale nie zosta� opatrzony numerem dzi�ki zastosowaniu polecenia \verb,\nonumber, (zamiast tego mo�na zastosowa� �rodowisko \texttt{equation*}), np.
\begin{equation*}
a^3-b^3=(a-b)(a^2+ab+b^2). 
\end{equation*}

Teraz zaprezentujemy odwo�ania do  numerowanych wzor�w. Wymaga to zwykle dwukrotnego przetworzenia tekstu �r�d�owego.

Na mocy wzoru (\ref{plus2}) mamy
\[
(x+3y)^2=x^2+6xy+9y^2,
\]
za� stosuj�c \eqref{minus2} dostajemy
\[
\left(3p-\frac{q}{2}\right)^2=9p^2-3pq+\frac{q^2}{4}.
\]
Wobec \eqref{roznica2} uzyskujemy
\[
(x-4)(x+4)=x^2-16.
\]

\end{document}

