\documentclass[12 pt, leqno]{article}
\usepackage[T1]{fontenc}
\usepackage[polish]{babel}
\usepackage[utf8]{inputenc}
\usepackage{amssymb, amsmath}

\author{Konrad Pagacz}

\begin{document}
\section{Ciągi}
Jeśli $(a_n)_{n=1,2,...}$ jest ciągiem arytmetycznym, to
\begin{equation}
S_n = \sum\limits_{k=1}^{n}a_k=\frac{a_1 + a_n}{2}n \label{suma}
\end{equation}

Stosując wzór \eqref{suma} otrzymujemy
\begin{equation}
1 + 2 + ... + n = \sum\limits_{k=1}^{n}k = \frac{n(n+1)}{2}. \label{suma-od1-don}
\end{equation}
W szczególności z \eqref{suma-od1-don} mamy
\begin{equation*}
1 + 2 + ... + 100 = \frac{100\cdot 101}{2} = 5050.
\end{equation*}

Jeśli $(a_n)_{n=1,2,3,...}$ jest ciągiem geometrycznym o ilorazie $q \ (q\neq 0)$, to
\begin{equation}
S_n=\sum\limits_{k=1}^{n}a_k=
\begin{cases}
a_1\frac{1-q^n}{1-q}, & \textnormal{gdy} \quad q\neq 1, \\
na_1, & \mbox{gdy} \quad q=1.
\end{cases} \label{suma-geom}
\end{equation}

Pyrzkładowo, wobec \eqref{suma-geom} dostajemy
\begin{equation*}
\sum\limits_{k=1}^{8}2^k=2+4+8+...+256=2\frac{1-2^8}{1-2}=2(2^8-1)=2\cdot 255=510.
\end{equation*}

\section{Macierze i układy równań}
Niech
\begin{equation}
A=\left[
\begin{array}{rrr}
2	&	1	&	3 \\
1	&	0	&	-4 \\
0	&	3	&	1
\end{array}
\right]
\label{macierzA}
\end{equation}

Macierz $A$ postaci \eqref{macierzA} jest macierzą główną układu równañ liniowych postaci
\begin{eqnarray*}
2x+y+3z	&	=	&	6, \\
x - 4z	&	=	&	-3, \\
3y + z	&	=	&	4,
\end{eqnarray*}
który można zapisać w formie bardziej przejrzystej
\begin{equation}
\begin{array}{rrrrrrr}
2x & + & y & + & 3z & = & 6 , \\
x &  &  & - & 4z & = & -3 , \\
 &  & 3y & + & z & = & 4 . 
\end{array}
\label{array-form}
\end{equation}

Zwykle układ równań zapisujemy z użyciem nawiasu klamrowego. Wtedy układ \eqref{array-form} wygląda następująco
\begin{equation*}
\left\{
\begin{array}{rrrrrrr}
2x & + & y & + & 3z & = & 6 , \\
x &  &  & - & 4z & = & -3 , \\
 &  & 3y & + & z & = & 4 
\end{array}
\right.
\end{equation*}
lub
\begin{equation*}
\left.
\begin{array}{rrrrrrr}
2x & + & y & + & 3z & = & 6 \\
x &  &  & - & 4z & = & -3  \\
 &  & 3y & + & z & = & 4 
\end{array}
\right\}.
\end{equation*}

\section{Pewna funkcja}
Rozważmy funkcję $g:\mathbb{R}\rightarrow\mathbb{R}$ postaci
\begin{equation*}
    g(x)=\left\{
    \begin{array}{ccc}
        \frac{x}{x-1} & \mbox{dla} & x < -1,  \\
        2\sin\!{(\pi x)} & \mbox{dla} &  x \in \langle -1,3), \\
        \frac{3x+1}{8x} & \mbox{dla} & x \in \langle 3,6 \rangle,  \\
        4x^3 & \mbox{dla} & x > 6.
    \end{array}
    \right.
\end{equation*}
\end{document}