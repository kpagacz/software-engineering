\documentclass[12pt]{beamer}

\usepackage[T1]{fontenc}
\usepackage[polish]{babel}
\usepackage[utf8]{inputenc}
\usetheme{Szeged}
\usepackage{amssymb, amsmath, multicol}
\usepackage{graphicx}
\setbeamertemplate{footline}[frame number]

\author{Konrad Pagacz}
\date{}
\title{Moja super hiper prezentacja}

\newtheorem{lm}{Lemat}

\begin{document}
% 1 slajd
\begin{frame}
\titlepage 
\end{frame}

% table of content
\begin{frame}
    \frametitle {Krok po kroku}
    \tableofcontents
\end{frame}

% slajd
\section{Krok 1.}
\begin{frame}
Rzeczy się tu dzieją na tym slajdzie.
\vspace{2cm}

\textit {Prawdą na zajęciach ze środowiska jest, że należy się zaangażować}
\end{frame}

\section{Teoria ślurpa}
\begin{frame}
    \frametitle{Wzór na ślurpa}
\begin{displaymath}
f(slurp) = 3*lenistwo + 7^2 + {\sum}' spanko
\end{displaymath}
\end{frame}

\begin{frame}
    \frametitle{Przykład ślurpa w naturalnym dla niego otoczeniu}
    \centering{\includegraphics[height=6cm]{spanko.jpg}}
\end{frame}

\section{Podstawowe prawa ślurpa}
\begin{frame}
    \frametitle{Fundamentalne prawo ślurpa}
    \begin{lm}
        Jeżeli trzeba coś zrobić, a można ślurpić, należy ślurpić.
    \end{lm}
\end{frame}

\section{Przybory ślurpa}
\begin{frame}
    \frametitle{Podstawowe wyposażenie ślurpa}
    \begin{enumerate}
        \item<1-> Łóżko
        \item<2-> Stream
        \item<3-> Nomy:
            \begin{enumerate}
                \item piciu
                \item żelki
            \end{enumerate}
    \end{enumerate}
\end{frame}

\begin{frame}
\textbf{To już jest koniec, czyż ślurpienie nie jest proste?}
\end{frame}


\end{document}