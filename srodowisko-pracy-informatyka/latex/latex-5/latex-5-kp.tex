\documentclass[12pt, a4paper]{article}

% polish language
\usepackage[T1]{fontenc}
\usepackage[polish]{babel}
\usepackage[utf8]{inputenc}

% packages
\usepackage{amssymb, amsmath, enumerate, multicol}
% indent first paragraph
\usepackage{indentfirst}


\begin{document}
\begin{center}
\Large{Wybrane zadania z analizy matematycznej}
\end{center}

Prezentujemy przykładowe zadania z zakresu analizy matematycznej, dotyczące funkcji dwóch zmiennych rzeczywistych. Wiele innych można znaleźć np. w \cite{banas}, \cite{krysicki}.

\begin{enumerate}
    \item Określić dziedznę nsatępujących funkcji i przedstawić ilustrację graficzną wyznaczonych zbiorów
        \begin{enumerate} [ {(}a{)} ]
        \begin{multicols}{2}
            \item $f(x,y)=\frac{\log\sqrt{9-4x^2-y^2}}{x-2},$
            \item $f(x,y) = \frac{\arccos{x+y}}{\arcsin{x}}.$
        \end{multicols}
        \end{enumerate}
    \item Zbadać istnienie poniższych granic
        \begin{enumerate}[ {(}a{)} ]
        \begin{multicols}{2}
            \item $\lim\limits_{(x,y)\rightarrow(3,4)}\frac{2-xy}{x^2+y^2},$
            \item $\lim\limits_{(x,y)\rightarrow(1,0)}\frac{\ln{(x+e^y)}}{\sqrt{x^2+y^2}},$
            \item $\lim\limits_{(x,y)\rightarrow(1,0)}\left(1+\frac{1}{x}\right)^{\frac{x^2}{x+y}},$
            \item $\lim\limits_{(x,y)\rightarrow(0,0)}\frac{xy}{2x^2 + y^2}.$
        \end{multicols}
        \end{enumerate}
    \item Zbadać ciągłość funkcji $f$ w punkcie $(0,0)$
    \begin{equation}
        f(x,y)=

            \frac{x^3 + y^3}{x^2 + 4y^2},   &\textrm{gdy} & \quad (x,y)\neq(0,0), \\
            1,  &   \textnormal{gdy}   &  \quad   x=y=0.

    \end{equation*}
\end{enumerate}

% bibliografia
\begin{thebibliography}{aa}
\bibitem{banas}J. Banaś, S. Wędrychowicz, \textit Zbiór zadań z analizy matematycznej
\bibitem{krysicki}W. Krysicki, L. Włodarski, \textit Analiza matematyczna w zadaniach
\end{thebibliography}

\end{document}